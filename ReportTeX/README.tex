\documentclass[letterpaper,11pt]{article}

\usepackage[utf8]{inputenc}
\usepackage[spanish,mexico]{babel}
\usepackage{graphicx}
\usepackage{amsmath}
\usepackage{amsthm}
\usepackage{svg}
\usepackage{amsfonts}
\usepackage{subcaption}
\usepackage[margin=1.5cm,
vmargin={1cm,0.3cm},
includefoot]{geometry}
\usepackage{fancyhdr}
\pagestyle{fancy}
\renewcommand{\headrulewidth}{0.4pt}
\renewcommand{\footrulewidth}{0.4pt}


\begin{document}
	\setlength{\unitlength}{1cm}
	\thispagestyle{empty}
	\begin{picture}(19,3)
	\put(-0.5,1.2){\includegraphics[scale=.20]{unam1.png}}
	\put(16,1){\includegraphics[scale=.29]{fciencias1.png}}
	\end{picture}

\begin{center}
\vspace{-114pt}
\textbf{\large Estructuras de Datos}\\
\textbf{ Semestre 2020-2}\\
Prof. Alejandro Hernández Mora\\
Ayud. Pablo Camacho González  \\ 
Ayud. Lab. Luis Manuel Martínez Dámaso   \\
\rule{17cm}{0.3mm}\\
\textbf{Simulador Supermercado}\\
\huge\textbf{Readme}\\[0.1cm]
\normalsize Kevin Ariel Merino Peña\footnote{317031326}\\
Armando Abraham Aquino Chapa\footnote{317058163}\\
\rule{17cm}{0.3mm}
\end{center}
%\vspace{-10pt}
%\begin{flushright}
%\vspace{-3pt}
%end{flushright}
\section*{Aclaraciones}
En el programa presentado se presentan dos modalidades de ejecución en el menú principal, la primera permite generar y cargar un documento para alimentar el almacén, resurtirlo o añadir algún producto y en la segunda opción (como usuario) se permite generar una simulación aleatoria (En realidad se realizan 14 simulaciones), cada simulación corresponde a un día, también ofrece la oportunidad de que el usuario eliga cuántas simulaciones va a realizar y que modifique en cada una el número de cajas rápidas o normales con las que se hará, al final de estas dos opciones se presenta la oportunidad de graficar las simulaciones que se hayan hecho, en un histograma y en una gráfica plana.
\section*{Instrucciones}
Para simplificar el proceso de compilado se propuso integrar el  código de la carpeta \textit{src} en el directorio \textbf{toExec/} para que al ingresar allí se ejecute el siguiente comando y el programa pueda ser probado
\begin{center}
	\textit{java -jar supermarket$\_$simulation.jar}
\end{center}
Cabe mencionar que se trata \textbf{del mismo código que se puede encontrar en src/} pero nos parece más real o al menos más cercano a una aplicación fidedigna de una entrega en el campo de la industria.


Esta carpeta cuenta con archivos \textit{.gp, .sh} que servirán para generar las gráficas con \textbf{gnuplot} mismo software que se requiere al menos en su versión \textit{5.0}
\end{document}
